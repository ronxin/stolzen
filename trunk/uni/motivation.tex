\documentclass[a4paper,14pt]{extarticle}

\usepackage{setspace}
\setstretch{1.3}

\usepackage{indentfirst}

\frenchspacing


\usepackage{cmap}

\usepackage{geometry}
\geometry{left=3cm}
\geometry{right=2cm}
\geometry{top=2cm}
\geometry{bottom=2cm}


\begin{document}

\section*{Motivation letter}

!!!
My name is Alexey Grigorev, I am 23 years old. As a graduate student I apply to the \b{Erasmus Mundus Master's Program in Information Technologies for Business Intelligence (IT4BI)} for the profound study of Business Intelligence and related aspects such as data warehousing, data mining and decision support. Through both studying and working I gained solid background, ...  and this program exactly matches my study objectives. Having good both academic and professional backgrounds, I am ... And this program ideally suits my interests and study objectives.


\subsection*{Academic background}

In 2006 entered Far Easter State Academy for Humanities and Social Studies (afterwards renamed to Priamurian State University) in the city of Birobidzan, Russia, to study computer science and in 2010 I graduated with honors and received bachelor's degree. The courses I took were chiefly devoted to database design for small enterprises: data management, theory of information processes and systems, corporate IT systems and so on. My final graduate thesis was on automated information system for enterprise \"Sterkh\", Birobidzhan. However, disciplines of the most interest for me were mathematical analysis, probability and statistics, and statical analysis of time series.

From the beginning of my studies I realized the importance of the English language and enrolled in an advances training of English and in 2011 graduated with a diploma, equated to bachelor's degree, which allows me to teach the language in general education institutions.

Apart from studies, I was actively engaged in many extracurricular activities. My main research interests during studies were software engineering, resulted in Von Neumann machine model shown in \"Programmer 2009\" conference; image processing, resulted in a paper about copyright protection and watermarking shown in \"Programmer 2010\" conference; and, lastly, applied statistics, with work around cluster analysis and time series procession shown in 57th Scientific Collegiate Conference in Far Eastern State University of Social Studies.

From 2007 to 2010 I annually was a part of PSU's team in Far Eastern Quarter Finals of North-eastern European Regional Collegiate Programming Contest held by ACM. In 2007 and 2008 I participated in individual mathematical contests, where in 2008 showed the best result amongst all PSU participants. Finally, In 2010 I competed in individual nation-wide programing contest where I ranked 4th among participants form Far East region of Russia.

!!!
My accomplishments did not go unnoticed and as a recognition in the end of 2009 I received a prestigious governmental award for my achievements in studying and extracurricular activities. However, among all the tournaments I have partaken in the most outstanding was \"the best graduate\" contest in my Alma Mater, between all PGU's graduates from all faculties - highly competent and competitive rivals, all with high chances of winning. But I managed to win by a narrow margin.


\subsection*{Professional background}

!!! 
Theory without practice costs little. That is why, while a student, I started my career as a programmer in Mathematical Simulation Laboratory at my university, where I worked on a system for predicting typhoons behavior. And as a part of my graduation thesis, I worked in Sterkh where I designed a system for controlling goods flow.

Upon graduation I decided to take a break from studying. There were two reasons for this: firstly, I wanted to acquire more practical knowledge, and secondly, I was driven by desire to find out what particular field is more interesting for me as well as more appealing for business. I moved to Nizhniy Novgorod where I worked in several companies as a software developer.

The first company was developing a system for robotizing e-commerce storages, and in the second one we were in charge of creating a system for selling ringback tones (melodies you hear in the background when you are calling to a mobile phone). The companies are quite different, but there was something that they have in common. Namely, they both need to process large volumes of data efficiently, quickly adapting to the needs of customers, be it putting complimentary goods on the same rack or suggesting a tone which a customer is likely to buy. 

!!!!
Now I am working in Poland at an investment bank. Trades from many exchanges across the globe have to be booked for commission calculation, and our team has to ensure that it is calculated correctly. Financial institutions have to deal with prodigious amounts of data, every day processing billion and billion of trades. It has once again proved that data analysis is crucial.


!!!!
I do my best to be not limited only by the scope of work and I am always looking forward to absorbing new knowledge. It means attending training, communities at work (architect community, agile community), local communities (like Nizhniy Novgorod Java community). At the moment I am undergoing a training about basics of data warehousing.

What is more, I have actively contributed to open source projects. For example, I played a team lead role in open source project jtalks.org, project for Russian-speaking Java community. Open source project have given me much: name, connections, experience in building enterprise-level application from scratch, plus experience in mentoring.

I understand the value of sharing knowledge and keep a professional blog where I post all my thoughts around IT, be it a review of a new technology,  summaries of books and technical articles, implementation of algorithms and so on. It started as a place where I could keep notes just for myself but eventually it evolved into a web site which around a hundred of people visit every day.

The main programming language I use in my every day activities is Java. However, I realize that it is important not to be bound to a certain technology and therefore I constantly broaden my mind by learning new tools, programming languages or different approaches. Recently I have started paying closer attention to the functional paradigm of programming, and I especially like the Scheme language for its conciseness and expressiveness. !!! not developed



\subsection*{Study objectives}

!!!!
It can be said that I have gained a significant insight into the needs of business. From my point of view, in the foreseeable future data scientists - programmers with statistics skills - will be in increasingly high demand. Enterprises have collected prodigious amounts of information over the last decades and each year the number of data is increased exponentially. Hence, data scientists will be needed desperately to put this data in order. 

While I am constantly self-improving and learning, it is not enough, and more solid educational background is needed. In the future I would like to concentrate on the following areas: business intelligence, data warehousing, data mining and big data. In order to be successful in this, I need a good intensive training or education, and such an outstanding program will definitely help me with it??.

Therefore I am deeply interested in the courses \b{IT4BI} provides because it fits perfectly my educational objectives. I am especially willing to develop decision support and data mining skills, and that is why I would like to choose \b{Decision Support and Business Intelligence} as a specialization. 
It will not only strengthen my knowledge in the field, but also help me fulfill ...

Additionally, I am highly interested in cultural exchange. I am always eager to learn about new cultures, traditions and cuisine. With this program, I will have an opportunity to live in Belgium and France and learn French. I am interested in these countries culture and art, especially in paiters such as Rene Magritte. Plus, architecture attracts me, and I will definitely will visit Bruges and will take a trip to see the castles of the Coire Valley. What is more, I am fond of Belgian chocolate. I like how French sounds, but unfortunately I know nothing about the language, and I am eager to learn it at least to... Therefore, I am looking forward to visiting both countries. 

Upon finishing the program I plan to pursue PhD and combine professional and teaching activities. The latter is especially important for me, because besides building my career I am extremely interested in sharing my experience and knowledge with others. The program definitely will give me a solid background for this. 

With my strong academic background, relevant working experience and constant crave for knowledge, I am confident that I am qualified and able to perform well in this program. 

Thank you very much for considering my application. I am looking forward to your positive response.

\end{document}


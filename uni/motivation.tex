\documentclass[a4paper,12pt]{article}

\usepackage{indentfirst}
\frenchspacing

\usepackage{cmap}

\usepackage{geometry}
\geometry{left=2cm}
\geometry{right=2cm}
\geometry{top=2cm}
\geometry{bottom=2cm}

\begin{document}

\section*{Motivation letter}

My name is Alexey Grigorev, and as a graduate student I apply to the \textbf{Erasmus Mundus Master's Program in Information Technologies for Business Intelligence (IT4BI)} for the profound study of Business Intelligence and related aspects such as data warehousing, data mining and decision support. I am fully qualified for the program, and with my strong academic and professional background I am a good match for the scholarship. Not only the curriculum is attractive for me, but the general idea of the Erasmus Mundus program of cultural exchange trough mobility and academic cooperation is also appealing.

\subsection*{Academic background}

In 2006 I entered Far Easter State Academy for Humanities and Social Studies (afterwards renamed to Priamurian State University) in the city of Birobidzan, Russia, to study computer science and in 2010 I graduated with honors and received bachelor's degree. The courses I took were chiefly devoted to database design for enterprises: data management, theory of information processes and systems, corporate IT systems and so on. My final graduate thesis was on automated information system for ``Sterkh'', Birobidzhan. However, disciplines of the most interest for me were mathematical analysis, probability and statistics and statical analysis of time series.

From the beginning of my studies I realized the importance of the English language and enrolled in an advances training of English and in 2011 graduated with a diploma, equated to a bachelor's degree, which allows me to teach the language in general education institutions.

Apart from studies, I was actively engaged in many extracurricular activities. My main research interests during studies were software engineering, resulted in Von Neumann machine model showed at ``Programmer 2009'' conference; image processing, resulted in a paper about copyright protection and watermarking showed at ``Programmer 2010'' conference; and, lastly, applied statistics, with work around cluster analysis and time series procession showed at 57th Scientific Collegiate Conference in Far Eastern State University of Social Studies.

From 2007 to 2010 I annually was a part of PSU's team in Far Eastern Quarter Finals of North-eastern European Regional Collegiate Programming Contest held by ACM. In 2007 and 2008 I participated in individual mathematical contests, where in 2008 I showed the best result amongst all PSU participants. Finally, in 2010 I competed in individual nation-wide programing contest where I ranked 4th among participants form Far Eastern region of Russia.

My achievements in studying and extracurricular activities did not go unnoticed and as a recognition in the end of 2009 I received a prestigious governmental award for talented youth.

However, among all the tournaments I have partaken in the most memorable was ``the best graduate'' contest in my Alma Mater between PSU's graduates from all faculties. I wanted to prove that I was able to succeed despite all difficulties. In the finals I faced highly competent and competitive rivals, all with high chances of taking the first place, but I persevered, and in the end I managed to win by a narrow margin. It showed that if I set a goal and work hard, I am able to achieve it.


\subsection*{Professional background}

Upon graduation I decided to take a break from studying. There were two reasons for this: firstly, I wanted to acquire more practical knowledge, and secondly, I was driven by desire to find out what particular field is more interesting for me as well as more appealing for business. I moved to Nizhniy Novgorod where I worked for several companies as a software developer.

The first company was developing a system for robotizing e-commerce storages, and in the second one we were responsible for creating a system for selling ringback tones (melodies you hear in the background when you are calling to a mobile phone). The companies are quite different, but there is something that they have in common. Namely, they both need to process large volumes of data efficiently, quickly adapting to the needs of customers, whether it be putting complimentary goods on the same rack or suggesting a tone which a customer is likely to buy.

Now I am working in Poland for an investment bank. Trades from many exchanges have to be processed and booked for commission calculation, and the goal of our team is to ensure that the resulting charge is accurate. I am able to see that financial institutions have to deal with large volume of information, every day processing billions and billions of trades. It also proves that data analysis is crucial, and now I realize that I want to concentrate further on this subject.

I do my best not to be limited only by the scope of work, and I have actively contributed to open source. I firmly believe in freedoms and benefits of open source software, and my most significant contribution was playing a team lead role in ``jtalks.org'' project. Its main goal was to enhance the underlying engine for the web-site of Russian-speaking Java community, but the more people joined the project, the more ideas appeared, and I was delighted to be at the epicenter of this development. Participation in open source is rewarding: it has given me connections, skills in building enterprise-level applications from scratch and experience in recruiting and mentoring.

What is more, I understand the value of sharing knowledge and keep a professional blog where I post all my thoughts around IT, be it a review of a new technology, summary of a book or a technical article and so on. It started as a place where I could keep notes just for myself, but eventually it evolved into a web site, which around a hundred of people visit every day. !!! Sometimes I receive requests for help from other java programmers, and for me it means that they believe in my experience and knowledge.

% With rapid development of new technologies, being limited only by the current scope of work means falling behind, and for this reason I am always looking forward to absorbing new knowledge. I attend trainings, take part in local communities (such as Nizhniy Novgorod Java community) or in communities at work (architect and agile community). At the moment, I am taking a training on data warehousing fundamentals and I find it very engaging.

\subsection*{Study objectives}

Trough my working experience I have gained a significant insight into the needs of business. From my point of view, in the not too distant future data scientists -- programmers with analytical skills and knowledge of statistics -- will be in increasingly high demand. The main reason for this is that over the last decades enterprises have accumulated prodigious amounts of information, and each year the number of data is doubled. Hence, such professionals will be needed desperately to put (this data) in order.

I am particularly eager to become a data scientists, and in the future I would like to focus on the following areas: business intelligence, data warehousing, data mining and big data. While I am constantly learning and self-improving, it is not enough, and I feel a certain lack of skills. For that reason, in order to be successful in these (fields), a good university education is needed, and such a remarkable master program will definitely help me reach my goals.

Therefore, I am highly motivated to take the courses \textbf{IT4BI} provides, because the curriculum fits perfectly my educational objectives. I am especially willing to develop decision support and data mining skills, and that is why I would like to choose Decision Support and Business Intelligence as a specialization.

Additionally, I am deeply interested in cultural exchange: traditions, art, and local cuisine (really) appeal to me. With this program, I will have an opportunity to live in Belgium and France, visit the Rene Magritte museum in Brussels and the Louvre in Paris. Architecture is my favorite form of art, and I will definitely explore Bruges in Belgium and will take a trip to see the castles of the Coire Valley in France. Most excitingly, it will give me such a great chance to learn French. I like how the language sounds, but unfortunately I have never had an opportunity to learn it, and I hope to reach at least lower-intermediate level during studies.

Upon finishing the program I plan to pursue PhD and combine professional and teaching activities. The qualification obtained through the studies will allow me to easily enroll in any doctoral program. I am specifically interested in researches conducted by Universit\'e Libre de Bruxelles and Pozna\'n University of Technology. Teaching is also important for me, because besides building my career I am enthusiastic about sharing my experience and knowledge with others. (!!!)

To summarize?, with my strong academic background, relevant working experience and constant crave for knowledge, I am confident that I am qualified and able to perform well in this program.

Thank you very much for considering my application. I am looking forward to your positive response.

\end{document}
